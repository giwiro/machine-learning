The main idea of linear regression is to predict some continuous value from input (\textit{x\textsubscript{i}}) using a function that has a linear behaviour ($\theta_0x_0 + \theta_1x_1 + \hdots + \theta_nx_n = \theta^Tx)$. The \textbf{goal} is to find the most accurate coefficients for the linear function we must reduce the error as much as possible, in other words, we gotta minimize the cost function.

\subsection{Types}
\subsubsection{Univariable}
When we have only 1 feature. The linear function for univariable regression should look like this:

\begin{align}
	h_{\theta}(x) & = \theta_{0} + \theta_{1}x
\end{align}

\subsubsection{Multivariable}
When we have more than 1 feature. The multivariable linear regression function should look like this:

\begin{align}
	h_{\theta}(x) & = \theta_{0}x_{0} + \theta_{1}x_{1} + \theta_{2}x_{2} + ... + \theta_{n}x_{n}
\end{align}

\subsection{Cost Function}
We can measure the accuracy of our hypotheses function by using a cost function. This takes an average difference (actually a fancier version of an average) of all the results of the hypotheses with inputs from x's and the actual output y's.

$$J(\theta) = \frac{1}{2m}\sum_{i=1}^{m} (h_{\theta}(x^{(i)}) - y^{(i)})^2$$

\subsection{Notation (Matrixes and Vector approaches)}
In order to make operations more code/cpu "friendly" we represent our set of data and variables on matrixes and vectors.

\begin{align*}
	x_j^{(i)} & = \text{value of feature } j \text{in the } i^{th} \text{ training example} \\
	x^{(i)} & = \text{the input (features) of the } i^{th} \text{ training example} \\
	m & = \text{The number of training examples} \\
	n & = \text{The number of features}
\end{align*}

\begin{align*}
    \theta_{(n \times 1)} & = \begin{bmatrix}
			\theta_0 \\
			\theta_1 \\
			\theta_2 \\
			\vdots \\
			\theta_{n} 
		\end{bmatrix}
	&
	x_{(n \times 1)} & = \begin{bmatrix}
			x_0 \\
			x_1 \\
			x_2 \\
			\vdots \\
			x_n 
		\end{bmatrix}
	&
	X_{(m \times n)} & = \renewcommand\arraystretch{1.2}\begin{bmatrix}
			x_0^1 & x_1^1 & x_2^1 \hdots x_n^1 \\
			x_0^2 & x_1^2 & x_2^2 \hdots x_n^2 \\
			x_0^3 & x_1^3 & x_2^3 \hdots x_n^3 \\
			\vdots & \vdots & \vdots \\
			x_0^m & x_1^m & x_2^m \hdots x_n^m \\
		\end{bmatrix}
	&
	y_{(m \times 1)} & = \begin{bmatrix}
			y_0 \\
			y_1 \\
			y_2 \\
			\vdots \\
			y_m
		\end{bmatrix}
\end{align*}

\begin{align*}
	\text{definition format: } h_{(\theta)} & = 
	\begin{bmatrix} \theta_0 & \theta_1 \hdots \theta_n \end{bmatrix}
	\begin{bmatrix} 
		x_0 \\
		x_1 \\
		\vdots \\
		x_n
	\end{bmatrix} = \theta^Tx
\end{align*}

\begin{align}
	\text{solution format: } h_{(\theta)} & = 
	\renewcommand\arraystretch{1.2}\begin{bmatrix}
			x_0^1 & x_1^1 & x_2^1 \hdots x_n^1 \\
			x_0^2 & x_1^2 & x_2^2 \hdots x_n^2 \\
			x_0^3 & x_1^3 & x_2^3 \hdots x_n^3 \\
			\vdots & \vdots & \vdots \\
			x_0^m & x_1^m & x_2^m \hdots x_n^m \\
		\end{bmatrix}
	\begin{bmatrix} \theta_0 & \theta_1 \hdots \theta_n \end{bmatrix} = X\theta
\end{align}

\begin{align}
	\text{cost function: } J(\theta) & = \frac{1}{2m}(X\theta - y)^2
\end{align}

\justify
\textbf{Note 1}: $x_0$ is a \textit{"hack"} and it's value is always 1. It's for making both ($\theta$ and $x$) same size in order to multiply them.
\justify
\textbf{Note 2}: The univariable and multivariable $h_{(\theta)}$ can be solved using $X\theta$


\subsection{How to obtain $\theta_{j}$}
The idea is to minimize the \textbf{Cost Dunction} (minimize the error). In order to do that, we need to build an equation by comparing the derivative of J to cero.

$$min(J(\theta)) = \frac{\partial{J(\theta)}}{\partial{\theta_{j}}} = \frac{1}{m}\sum_{i=1}^{m} (h_{\theta}(x^{(i)}) - y^{(i)})x_{j}^{(i)} = 0$$
We got 2 ways to approach this problem:

\begin{enumerate}[label=\arabic*.]
\item \textit{Gradient Descent}
\item \textit{Normal Equation}
\end{enumerate}

\subsubsection{Gradient Descent}
It is an iterative algorithm that, with a fixed learning rate ($\alpha$), updates all values of $\theta$ simultaneously and decreases the Cost Function.
The learning rate ($\alpha$) is adjusted by try/error. The initial values to test should be on a log-scale, at multiplicative
steps of about 3 times the previous value (i.e., 0.3, 0.1, 0.03, 0.01 and so on).
\par
\begin{center}
repeat until converge \{
	$$\theta_{j} = \theta_{j} - \alpha \frac{\partial{J(\theta)}}{\partial{\theta_j}}$$
\}
\end{center}

Since it's an iterative algorithm we want the steps to be really small in order to make the computation time a lot faster. We care about the relationship between entries, not the actual values; that why we tend to make the mean zero. For that we got 2 methods:

\begin{enumerate}[label=\arabic*.]
\item 
	\textit{Feature Scaling}: Involves dividing the input values by the range (i.e. the maximum value minus the minimum value) of the input variable, resulting in a new range of just 1.

\item 
	\textit{Mean normalization}: Involves subtracting the average value for an input variable from the values for that input variable resulting in a new average value for the input variable of just zero.

	$$x_i = \frac{x_i - \mu_i}{s_i}$$

	Where $\mu_i$ is the average of all the values for feature $(i)$ and $s_i$ is the range of values (max - min), or $s_i$ is the standard deviation.

	Note that dividing by the range, or dividing by the standard deviation, give different results. 
\end{enumerate}

\subsubsection{Normal Equation}
This is the analytical way to tackle this problem. We actually solve the equation $\frac{\partial{J(\theta)}}{\partial{\theta_{j}}} = 0$. There is no need to do feature scaling with the normal equation.

$$\theta = (X^TX)^{-1}X^Ty$$


\subsubsection{Comparison}
The main differences between this two approaches:


\begin{center}
\begin{tabular}{ |c|c| } 
\hline
\textbf{Gradient Descent} & \textbf{Normal Equation} \\
\hline
Need to choose alpha & No need to choose alpha \\
\hline
Needs many iterations & No need to iterate \\
\hline
$O(kn^2)$ & $O(n^3)$, need to calculate inverse of $X^TX$ \\
\hline
Works well when $n$ is large & Slow if $n$ is very large \\
\hline
\end{tabular}
\end{center}

